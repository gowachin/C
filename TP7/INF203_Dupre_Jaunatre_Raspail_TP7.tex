\documentclass[12pt,a4paper,notitlepage,colorinlistoftodos]{article}
%%%%%%%%%%%%%%%%%%%%%%%%%%%%%%%%%%%%%%%%%%%%%%%%%%%%%%%%%%%%%%%%%%%%%
% Template pour rendus Master
%%%%%%%%%%%%%%%%%%%%%%%%%%%%%%%%%%%%%%%%%%%%%%%%%%%%%%%%%%%%%%%%%%%%%%
\usepackage[utf8]{inputenc} %encodage
\usepackage[T1]{fontenc}

\usepackage[square,sort,comma,numbers]{natbib} % bibliography
\setcitestyle{authoryear,open={(},close={)}}
\renewcommand{\bibsection}{}

\usepackage[french]{babel} % langue
% add '\-' to create custom hypernation if a word is difficult to cut or :
\hyphenation{geo-graphique}

%mise en page générale
\usepackage{geometry}
%\geometry{a4paper} % format de feuille
\geometry{top=2.5cm, bottom=2.5cm, left=2.5cm, right=2.5cm} %marges
\usepackage{mathptmx} % Police Times si compilateur pdfLatex
\usepackage{amsmath,amsthm,amssymb}
\usepackage{times}
\setlength{\parindent}{0pt}

\linespread{1} % interligne
\usepackage{fancyhdr} %en tete et pied de page
\usepackage{lastpage}  %marche pas chez julia
\pagestyle{plain} 

\usepackage{lscape} % page en landscape

\usepackage{multicol}
\usepackage{hhline}
\setlength{\columnsep}{1cm}

\usepackage{hyperref,url} % lien cliquables
\hypersetup{
colorlinks = true,
linkcolor = black,
urlcolor = RoyalBlue
}
\usepackage{lipsum} %Lorem ipsum

\usepackage{wrapfig} %position d'images dans le texte
\usepackage{graphicx, subcaption, setspace, booktabs, wrapfig}

\usepackage[table,dvipsnames]{xcolor}

\usepackage{caption}
\DeclareCaptionType{annexe}[Annexe][Liste d'annexes] % rajout pour captions annexes
\DeclareCaptionType{web}[Web][Sites Web] % rajout pour mettre des captions web

\usepackage{todonotes} % notes et commentaires
%\usepackage[disable]{todonotes} % pour supprimer les commentaires lors de la compil

\usepackage[export]{adjustbox}

\usepackage[para,online,flushleft]{threeparttable}


\usepackage{listings}
\usepackage{color}
 
%http://latexcolor.com/ 
\definecolor{codegray}{rgb}{0.5,0.5,0.5}
\definecolor{cerulean}{rgb}{0.0, 0.48, 0.65}
\definecolor{beaublue}{rgb}{0.95, 0.95, 0.95}
\definecolor{amber}{rgb}{1.0, 0.25, 0.0}
\definecolor{indiagreen}{rgb}{0.07, 0.53, 0.03}
\definecolor{number}{rgb}{0.01, 0.01, 0.01}


\lstset{language=C}

  \lstset{emphstyle=\color{blue},
  inputencoding=latin1,
  basicstyle=\footnotesize,
  breaklines=true,
  keywordstyle=\bf\color{amber},
  commentstyle=\color{indiagreen},
  stringstyle=\color{cerulean},
  numberstyle=\color{number},
  backgroundcolor=\color{beaublue},
  morecomment=[s][\color{black}]{[}{]},
  morecomment=[s][\color{cerulean}]{[~}{]:},
  numbers=none,
  numbersep=5pt,
  lineskip=0.7pt,
  columns=fullflexible, % was flexible before but it induce space in '/2entier' like '/2 entier'
  showstringspaces=false ,
  literate=%
         {é}{{\'e}}1}
        
          \newcommand{\FSource}[1]{%
          \lstinputlisting[texcl=true]{#1}
          }

%%%%%%%%%%%% skip an all paragraphe, between this bornes %%%%%%%%%%%%%%%%%%
%\iffalse
%\fi

%%%%%%%%%%%%%%%%%%%%%%%%%%% nouvelles commandes spécifique au doc %%%%%%%%%%%%%%%%%%

\DeclareRobustCommand{\rchi}{{\mathpalette\irchi\relax}}
\newcommand{\irchi}[2]{\raisebox{\depth}{$#1\chi$}}
\renewcommand*\contentsname{Table des matières}
\newcommand{\unit}[1]{\hfill\text{}[\mathrm{#1}]}
%%%%%%%%%%%%%%%%%%%%%%%%%%%%%%%%%%%%%%%%%%%%%%%%%%%%%%%%%%%%%%%%%%%%%%
% Page de garde
%%%%%%%%%%%%%%%%%%%%%%%%%%%%%%%%%%%%%%%%%%%%%%%%%%%%%%%%%%%%%%%%%%%%%%
\begin{document}

\begin{figure}
    \begin{minipage}{.75\textwidth}
    \begin{center}
    {\Large INF203 : Compte-rendu TP7 \\ \textbf{Arguments de la ligne de commande et C
Entrées-sorties fichiers - Fichiers standard}}
    \end{center}
    %\vspace{\baselineskip}
    %\setlength{\parskip}{\smallskipamount}
    \rule{7em}{.4pt}\par
     Alexandre Dupré, Maxime Jaunatre, Clément Raspail | INF - 3 \par 
     %\href{mailto:\href{mailto:maxime.jaunatre@etu.univ-grenoble-alpes.fr}{Mail $^1$} | \today \par 
     \href{mailto:alexandre.dupre@etu.univ-grenoble-alpes.fr,maxime.jaunatre@etu.univ-grenoble-alpes.fr, clement.raspail@etu.univ-grenoble-alpes.fr}{Mail} | \today
\end{minipage}
\end{figure}

\hrule

\section*{Syntaxe}

\iffalse
 Alexandre Dupré <alexandre.dupre@etu.univ-grenoble-alpes.fr>
 Maxime Jaunatre <maxime.jaunatre@etu.univ-grenoble-alpes.fr> 
 Clément Raspail <clement.raspail@etu.univ-grenoble-alpes.fr>
\fi

Pour ce compte rendu la syntaxe des commandes sera la suivante :
\begin{lstlisting}
[~chemin]: commande
retour de la commande
\end{lstlisting}

Example :
\begin{lstlisting}
[~/INF203]: ls
sauve_TP1   TP1   TP2
\end{lstlisting}

Si la commande est interactive et demande d'appuyer  sur entrée, une caractère '->' est indiqué.

Les fichiers sont en \textit{italique} et les commandes (ou détails de retour de commande) en \textbf{gras}. Un script sera donc en gras quand il sera appelé comme une commande.


Les fichiers sources en C sont compilés avec clang, et un nom est donné avec l'option \textbf{clang -o}. Cela implique que les programmes seront appelés par un autre nom que \textbf{a.out}.

% #################################################################################################################################################################

\subsection*{Pour s'échauffer : quelques gammes avec des chaînes de caractères}


[a] 
\textbf{argc} represente le \$\# de shell + 1 soit le nombre d'agument donné en parametre, plus le nom du fichier.

\begin{lstlisting}[numbers=left, firstnumber = 0 ]
[~/INF203/TP7]: cat somme.c
#include<stdio.h>

// print arguments with + between and = at the end of the printf
int main (int argc, char *argv[]) {
  int i ;
  for (i=1 ; i<argc ; i++){
    printf("%s ", argv[i]) ;
    if(i != argc-1)
      printf("+ ");
  }
  printf("=");
  printf("\n");
  return 0;
}
\end{lstlisting}


[b]
Quand on ne met pas d'argument la commande renvoie un egale car on l'affiche a la fin de la boucle donc tout le temps.

[c]
\begin{lstlisting}[numbers=left, firstnumber = 0 ]
[~/INF203/TP7]: cat somme.c
#include<stdio.h>

int char2int(char c[]){
  int val = 0;
  val = val + c[0] - '0';
  for(int i = 1 ; c[i] != '\0' ; i++){
        val = val * 10 + (c[i] - '0');
    }
  return val;
}

int main (int argc, char *argv[]) {
  int i ;
  int s = 0 ;
  for (i=1 ; i<argc ; i++){
    s = s + char2int(argv[i]);
    printf("%s ", argv[i]) ;
    if(i != argc-1)
      printf("+ ");
  }
  printf("= %d", s);
  printf("\n");
  return 0;
}
\end{lstlisting}

\begin{lstlisting}
[~/INF203/TP7]: clang -o somme somme.c ; ./somme 29 10
29 + 10 = 39
[~/INF203/TP7]: clang -o somme somme.c ; ./somme 
= 0
\end{lstlisting}


[d] [e]
\begin{lstlisting}
[~/INF203/TP7]: clang -o mes_entrees_sorties mes_entrees_sorties.c
[~/INF203/TP7]: ./mes_entrees_sorties Candide_chapitre1.txt truc.txt
Rentrez un nombre correct d'arguments
[~/INF203/TP7]: ./mes_entrees_sorties truc.txt
truc.txt n'a pas pu être ouvert en lecture
[~/INF203/TP7]: ls -l Candide_chapitre1.txt 
-rw-r--r-- 1 maxime maxime 4992 mars  23 09:01 Candide_chapitre1.txt
[~/INF203/TP7]: chmod u-r Candide_chapitre1.txt 
[~/INF203/TP7]: ls -l Candide_chapitre1.txt 
--w-r--r-- 1 maxime maxime 4992 mars  23 09:01 Candide_chapitre1.txt
[~/INF203/TP7]: clang -o mes_entrees_sorties mes_entrees_sorties.c ; ./mes_entrees_sorties Candide_chapitre1.txt 
Candide_chapitre1.txt n'a pas pu être ouvert en lecture
\end{lstlisting}


[f]
\begin{lstlisting}[numbers=left, firstnumber = 0 ]
[~/INF203/TP7]: cat mes_entrees_sorties.c
#include <stdio.h>

int main (int argc, char *argv[]) {
	FILE *f ;
	FILE *g ;
	char c ;
	if(argc == 2){
		f = fopen(argv[1], "r") ;
		if (f == NULL) {
			printf("%s n'a pas pu être ouvert en lecture\n", argv[1]) ;
		return 2 ;
		}
	
		fscanf(f, "%c", &c) ;
		while (!feof(f)) {
			printf("%c", c) ;	
			fscanf(f, "%c", &c) ;
		}
		printf("\n");
	} else if (argc == 3) {
		f = fopen(argv[1], "r") ;
		g = fopen(argv[2], "a") ;
		if (f == NULL) {
			printf("%s n'a pas pu être ouvert en lecture\n", argv[1]) ;
			return 2 ;
		}
		if (g == NULL) {
			printf("%s n'a pas pu être ouvert en lecture\n", argv[2]) ;
			return 2 ;
		}
		fprintf(g, "%c", '\n');
		fscanf(f, "%c", &c);
		while (!feof(f)) {
			fprintf(g,"%c", c) ;	
			fscanf(f, "%c", &c) ;
		}

	} else { 
		printf("Rentrez un nombre correct d'arguments (1) \n");
		return 1 ;
	}
	fclose(f) ;
	return 0 ;
}
\end{lstlisting}

[g] [h]
\textbf{cat *.c} affiche dans le terminal le contenu de tout les fichiers dont l'extention est \textit{.c}.
\textbf{cat} sans arguments entre dans une boucle while attendant une entrée standard pour l'afficher dans le terminal après. 
Il faut entrer la combinaison de touches \textit{ctrl-d} pour sortir de cette boucle.

[i]
\begin{lstlisting}[numbers=left, firstnumber = 0 ]
[~/INF203/TP7]: ./mon_cat mon_cat.c
#include <stdio.h>

int main (int argc, char *argv[]) {
	FILE *f ;
	char c ;
	int i ;
	if (argc == 1){
		char c;
		fscanf(stdin, "%s", &c);
		while( !feof(stdin)){
			printf("%s\n", &c);
			fscanf(stdin, "%s", &c);
		}
		return 0;
	}

	for(i = 1; i <argc ; i++ ){
		//printf("l'argument numero %d est %s\n",i, argv[i]) ;
		f = fopen(argv[i], "r") ;
		if (f == NULL) {
			printf("%s n'a pas pu être ouvert en lecture\n", argv[i]) ;
		return 2 ;
		}
	
		fscanf(f, "%c", &c) ;
		while (!feof(f)) {
			printf("%c", c) ;	
			fscanf(f, "%c", &c) ;
		}
		printf("\n");
		fclose(f) ;
	} 
	return 0 ;
}
\end{lstlisting}


[j]
\begin{lstlisting}[numbers=left, firstnumber = 0 ]
[~/INF203/TP7]: cat occurences.c
#include <stdio.h>

int main(int argc, char *argv[]){
    FILE* f;
    int list[128];
    char c;
    // init list to 0
    for (int i = 0; i < 128; i++){
        list[i] = 0;
    }
    // open file
    f = fopen( argv[1], "r");
    if ( f == NULL){
        printf("Rentrez un nom de ficher");
        return 1;
    }
    // scan file
    fscanf(f, "%c", &c);
    list[c]++;
    while(!feof(f)){
        fscanf(f, "%c", &c);
        list[c]++;
    }
    // print value
    for (int i = 32; i < 127; i++){ 
        // on évite les caractères spéciaux (saut de ligne) et DEL
        if(list[i] > 0){
            printf("Il y a %d occurences du caractère '%c'\n", list[i], i);
        }
    }
    
    return 0;
}
\end{lstlisting}

\begin{lstlisting}
[~/INF203/TP7]: ./mon_cat test.txt 
Never gonna give you up
Never let you down
[~/INF203/TP7]: ./occurences test.txt 
Il y a 7 occurences du caractère ' '
Il y a 2 occurences du caractère 'N'
Il y a 1 occurences du caractère 'a'
Il y a 1 occurences du caractère 'd'
Il y a 6 occurences du caractère 'e'
Il y a 2 occurences du caractère 'g'
Il y a 1 occurences du caractère 'i'
Il y a 1 occurences du caractère 'l'
Il y a 4 occurences du caractère 'n'
Il y a 4 occurences du caractère 'o'
Il y a 1 occurences du caractère 'p'
Il y a 2 occurences du caractère 'r'
Il y a 1 occurences du caractère 't'
Il y a 3 occurences du caractère 'u'
Il y a 3 occurences du caractère 'v'
Il y a 1 occurences du caractère 'w'
Il y a 2 occurences du caractère 'y'
\end{lstlisting}


%%%%%%%%%%%%%%%%%%%%%%%%%%%%%%%%%%%%%%%%%%%%%%%%%%%%%%%%%%%%%%%%%%%%%%
% Références
%%%%%%%%%%%%%%%%%%%%%%%%%%%%%%%%%%%%%%%%%%%%%%%%%%%%%%%%%%%%%%%%%%%%%%
%\newpage
\cite{}
%\subsection*{Bibliographie}
\bibliographystyle{authordate1}
\bibliography{ICU}

\end{document}