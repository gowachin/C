\documentclass[12pt,a4paper,notitlepage,colorinlistoftodos]{article}
%%%%%%%%%%%%%%%%%%%%%%%%%%%%%%%%%%%%%%%%%%%%%%%%%%%%%%%%%%%%%%%%%%%%%
% Template pour rendus Master
%%%%%%%%%%%%%%%%%%%%%%%%%%%%%%%%%%%%%%%%%%%%%%%%%%%%%%%%%%%%%%%%%%%%%%
\usepackage[utf8]{inputenc} %encodage
\usepackage[T1]{fontenc}

\usepackage[square,sort,comma,numbers]{natbib} % bibliography
\setcitestyle{authoryear,open={(},close={)}}
\renewcommand{\bibsection}{}

\usepackage[french]{babel} % langue
% add '\-' to create custom hypernation if a word is difficult to cut or :
\hyphenation{geo-graphique}

%mise en page générale
\usepackage{geometry}
%\geometry{a4paper} % format de feuille
\geometry{top=2.5cm, bottom=2.5cm, left=2.5cm, right=2.5cm} %marges
\usepackage{mathptmx} % Police Times si compilateur pdfLatex
\usepackage{amsmath,amsthm,amssymb}
\usepackage{times}
\setlength{\parindent}{0pt}

\linespread{1} % interligne
\usepackage{fancyhdr} %en tete et pied de page
\usepackage{lastpage}  %marche pas chez julia
\pagestyle{plain} 

\usepackage{lscape} % page en landscape

\usepackage{multicol, multirow}
\usepackage{hhline}
\setlength{\columnsep}{1cm}

\usepackage{hyperref,url} % lien cliquables
\hypersetup{
colorlinks = true,
linkcolor = black,
urlcolor = RoyalBlue
}
\usepackage{lipsum} %Lorem ipsum

\usepackage{wrapfig} %position d'images dans le texte
\usepackage{graphicx, subcaption, setspace, booktabs, wrapfig}

\usepackage[table,dvipsnames]{xcolor}

\usepackage{caption}
\DeclareCaptionType{annexe}[Annexe][Liste d'annexes] % rajout pour captions annexes
\DeclareCaptionType{web}[Web][Sites Web] % rajout pour mettre des captions web

\usepackage{todonotes} % notes et commentaires
%\usepackage[disable]{todonotes} % pour supprimer les commentaires lors de la compil

\usepackage[export]{adjustbox}

\usepackage[para,online,flushleft]{threeparttable}


\usepackage{listings}
\usepackage{color}
 
%http://latexcolor.com/ 
\definecolor{codegray}{rgb}{0.5,0.5,0.5}
\definecolor{cerulean}{rgb}{0.0, 0.48, 0.65}
\definecolor{beaublue}{rgb}{0.95, 0.95, 0.95}
\definecolor{amber}{rgb}{1.0, 0.25, 0.0}
\definecolor{indiagreen}{rgb}{0.07, 0.53, 0.03}
\definecolor{number}{rgb}{0.01, 0.01, 0.01}


\lstset{language=C}

  \lstset{emphstyle=\color{blue},
  inputencoding=latin1,
  basicstyle=\footnotesize,
  breaklines=true,
  keywordstyle=\bf\color{amber},
  commentstyle=\color{indiagreen},
  stringstyle=\color{cerulean},
  numberstyle=\color{number},
  backgroundcolor=\color{beaublue},
  morecomment=[s][\color{black}]{[}{]},
  morecomment=[s][\color{cerulean}]{[~}{]:},
  numbers=none,
  numbersep=5pt,
  lineskip=0.7pt,
  columns=fullflexible, % was flexible before but it induce space in '/2entier' like '/2 entier'
  showstringspaces=false ,
  literate=%
         {ê}{{\^e}}1
         {é}{{\'e}}1}
        
          \newcommand{\FSource}[1]{%
          \lstinputlisting[texcl=true]{#1}
          }
          
\lstdefinelanguage{rshell}{
	morecomment=[s][\color{cerulean}]{[~}{]:},
}

%\lstdefinestyle{numbers} {numbers=left, stepnumber=1, numberstyle=\tiny, numbersep=10pt}
%\lstdefinestyle{MyFrame}{backgroundcolor=\color{yellow},frame=shadowbox}
%
%\lstdefinestyle{MyCStyle} {language=C,style=numbers,style=MyFrame,frame=lines}
%\lstdefinestyle{MyC++Style} {language=C++,style=numbers,style=MyFrame,frame=none,backgroundcolor={}}
%
%\lstset{language=C,frame=lines}
%\lstset{language=C++,frame=none}


%%%%%%%%%%%% skip an all paragraphe, between this bornes %%%%%%%%%%%%%%%%%%
%\iffalse
%\fi

%%%%%%%%%%%%%%%%%%%%%%%%%%% nouvelles commandes spécifique au doc %%%%%%%%%%%%%%%%%%

\DeclareRobustCommand{\rchi}{{\mathpalette\irchi\relax}}
\newcommand{\irchi}[2]{\raisebox{\depth}{$#1\chi$}}
\renewcommand*\contentsname{Table des matières}
\newcommand{\unit}[1]{\hfill\text{}[\mathrm{#1}]}
%%%%%%%%%%%%%%%%%%%%%%%%%%%%%%%%%%%%%%%%%%%%%%%%%%%%%%%%%%%%%%%%%%%%%%
% Page de garde
%%%%%%%%%%%%%%%%%%%%%%%%%%%%%%%%%%%%%%%%%%%%%%%%%%%%%%%%%%%%%%%%%%%%%%
\begin{document}

\begin{figure}
    \begin{minipage}{.75\textwidth}
    \begin{center}
    {\Large INF203 : Compte-rendu TP11 \\ \textbf{Encore des automates ...}}
    \end{center}
    %\vspace{\baselineskip}
    %\setlength{\parskip}{\smallskipamount}
    \rule{7em}{.4pt}\par
     Alexandre Dupré, Maxime Jaunatre, Clément Raspail | INF - 3 \par 
     %\href{mailto:\href{mailto:maxime.jaunatre@etu.univ-grenoble-alpes.fr}{Mail $^1$} | \today \par 
     \href{mailto:alexandre.dupre@etu.univ-grenoble-alpes.fr,maxime.jaunatre@etu.univ-grenoble-alpes.fr, clement.raspail@etu.univ-grenoble-alpes.fr}{Mail} | \today
\end{minipage}https://prod.liveshare.vsengsaas.visualstudio.com/join?14D93CF16727CB7DF166E1D544FCB769905A
\end{figure}

\hrule

\section*{Syntaxe}

\iffalse
 Alexandre Dupré <alexandre.dupre@etu.univ-grenoble-alpes.fr>
 Maxime Jaunatre <maxime.jaunatre@etu.univ-grenoble-alpes.fr> 
 Clément Raspail <clement.raspail@etu.univ-grenoble-alpes.fr>
\fi

Pour ce compte rendu la syntaxe des commandes sera la suivante :
\begin{lstlisting}
[~chemin]: commande
retour de la commande
\end{lstlisting}

Example :
\begin{lstlisting}
[~/INF203]: ls
sauve_TP1   TP1   TP2
\end{lstlisting}

Si la commande est interactive et demande d'appuyer  sur entrée, une caractère '->' est indiqué.

Les fichiers sont en \textit{italique} et les commandes (ou détails de retour de commande) en \textbf{gras}. Un script sera donc en gras quand il sera appelé comme une commande.


Les fichiers sources en C sont compilés avec clang, et un nom est donné avec l'option \textbf{clang -o}. Cela implique que les programmes seront appelés par un autre nom que \textbf{a.out}.

% #################################################################################################################################################################

\section{Boucle de lecture}

[a]
Dans un premier temps \textbf{executer_ligne_commande} remplace les variables, s'il y en a, par leurs valeurs. 
S'il n'y a pas de variables elle test si il y a une affectation de commande. 
Sinon, si c'est c'est une commande externe, alors renvoie le résultat de la commande.


[b]
Pour l'instant \textbf{lire_ligne_fichier} ne fait rien, \textbf{appliquer_expansion_variables} copie la ligne expensé dans la ligne
originale, \textbf{decouper_ligne} associe a un tableau la ligne et l'élément NULL.


\section{Lecture d'une ligne}

[c]
\begin{lstlisting}[numbers=left, firstnumber = 30]
[~/INF203/TP11]: tail -n 42 lignes.c | head -n 22
int lire_ligne_fichier(FILE * fichier, char *ligne) {
/*************** A COMPLETER ******************/
    /*
       Lecture d'une ligne depuis un fichier avec stockage de la ligne lue
       dans la variable ligne. Ne pas inclure le '\n' final dans la variable
       ligne, ne pas oublier de completer la ligne avec un '\0' terminal.
       La valeur de retour doit etre 0 s'il n'y a plus rien a lire (fin de
       fichier sans avoir lu aucun caractere) et 1 sinon.
     */
/**********************************************/
    char c;
    strcpy(ligne, "");
    fscanf(fichier, "%c", &c);
    while ((c != '\n') && (!feof(fichier))) {
        strncat(ligne, &c, 1);
        fscanf(fichier, "%c", &c) ;   
    }
    strncat(ligne, "\0", 1);
    if (feof(fichier))
        return 0;
    return 1;
}
\end{lstlisting}


[d]
\begin{lstlisting}[language = rshell]
[~/INF203/TP11]: ./interpreteur test_1_lecture_ligne.sh
Ligne lue : echo Affichage avec arguments, qui ne marche pas au debut
J'execute la commande : echo Affichage avec arguments, qui ne marche pas au debut
Avec les arguments :
echo Affichage avec arguments, qui ne marche pas au debut: No such file or directory
Code de retour de la commande : 65280
Ligne lue : pwd
J'execute la commande : pwd
Avec les arguments :
/home/maxime/git/C/TP11
Code de retour de la commande : 0
Ligne lue : ls
J'execute la commande : ls
Avec les arguments :
debug.h                                 interpreteur.c  lignes.h        main_test_vars.c  systeme.c  test_1_lecture_ligne.sh  TP11.pdf          variables.h
INF203_Dupre_Jaunatre_Raspail_TP11.tex  interpreteur.o  lignes.o        main_test_vars.o  systeme.h  test_2_variables.sh      variables_base.h  variables.o
interpreteur                            lignes.c        main_test_vars  Makefile          systeme.o  test_3_automatique.sh    variables.c
Code de retour de la commande : 0
Ligne lue : uptime
J'execute la commande : uptime
Avec les arguments :
 09:23:01 up  1:18,  1 user,  load average: 1,68, 1,95, 2,16
Code de retour de la commande : 0
Ligne lue : pwd
J'execute la commande : pwd
Avec les arguments :
/home/maxime/git/C/TP11
Code de retour de la commande : 0
Ligne lue : ls
J'execute la commande : ls
Avec les arguments :
debug.h                                 interpreteur.c  lignes.h        main_test_vars.c  systeme.c  test_1_lecture_ligne.sh  TP11.pdf          variables.h
INF203_Dupre_Jaunatre_Raspail_TP11.tex  interpreteur.o  lignes.o        main_test_vars.o  systeme.h  test_2_variables.sh      variables_base.h  variables.o
interpreteur                            lignes.c        main_test_vars  Makefile          systeme.o  test_3_automatique.sh    variables.c
Code de retour de la commande : 0
Ligne lue : uptime
J'execute la commande : uptime
Avec les arguments :
 09:23:01 up  1:18,  1 user,  load average: 1,68, 1,95, 2,16
Code de retour de la commande : 0
\end{lstlisting}

On constate que les commandes qui ont marché sont affichées en double.


On a pus effectuer toute les fonctions ainsi que les tester, tout fonctionne correctement. Les 3 fichiers de test sont passé
ainsi que des tests suplémentaire. 
\textit{On ne met pas tout les morceaux de code ici car il y a plus de 200 lignes, et dans un pdf ce n'est pas pratique.}


%%%%%%%%%%%%%%%%%%%%%%%%%%%%%%%%%%%%%%%%%%%%%%%%%%%%%%%%%%%%%%%%%%%%%%
% Références
%%%%%%%%%%%%%%%%%%%%%%%%%%%%%%%%%%%%%%%%%%%%%%%%%%%%%%%%%%%%%%%%%%%%%%
%\newpage
\cite{}
%\subsection*{Bibliographie}
\bibliographystyle{authordate1}
\bibliography{ICU}

\end{document}