\documentclass[12pt,a4paper,notitlepage,colorinlistoftodos]{article}
%%%%%%%%%%%%%%%%%%%%%%%%%%%%%%%%%%%%%%%%%%%%%%%%%%%%%%%%%%%%%%%%%%%%%
% Template pour rendus Master
%%%%%%%%%%%%%%%%%%%%%%%%%%%%%%%%%%%%%%%%%%%%%%%%%%%%%%%%%%%%%%%%%%%%%%
\usepackage[utf8]{inputenc} %encodage
\usepackage[T1]{fontenc}

\usepackage[square,sort,comma,numbers]{natbib} % bibliography
\setcitestyle{authoryear,open={(},close={)}}
\renewcommand{\bibsection}{}

\usepackage[french]{babel} % langue
% add '\-' to create custom hypernation if a word is difficult to cut or :
\hyphenation{geo-graphique}

%mise en page générale
\usepackage{geometry}
%\geometry{a4paper} % format de feuille
\geometry{top=2.5cm, bottom=2.5cm, left=2.5cm, right=2.5cm} %marges
\usepackage{mathptmx} % Police Times si compilateur pdfLatex
\usepackage{amsmath,amsthm,amssymb}
\usepackage{times}
\setlength{\parindent}{0pt}

\linespread{1} % interligne
\usepackage{fancyhdr} %en tete et pied de page
\usepackage{lastpage}  %marche pas chez julia
\pagestyle{plain} 

\usepackage{lscape} % page en landscape

\usepackage{multicol}
\usepackage{hhline}
\setlength{\columnsep}{1cm}

\usepackage{hyperref,url} % lien cliquables
\hypersetup{
colorlinks = true,
linkcolor = black,
urlcolor = RoyalBlue
}
\usepackage{lipsum} %Lorem ipsum

\usepackage{wrapfig} %position d'images dans le texte
\usepackage{graphicx, subcaption, setspace, booktabs, wrapfig}

\usepackage[table,dvipsnames]{xcolor}

\usepackage{caption}
\DeclareCaptionType{annexe}[Annexe][Liste d'annexes] % rajout pour captions annexes
\DeclareCaptionType{web}[Web][Sites Web] % rajout pour mettre des captions web

\usepackage{todonotes} % notes et commentaires
%\usepackage[disable]{todonotes} % pour supprimer les commentaires lors de la compil

\usepackage[export]{adjustbox}

\usepackage[para,online,flushleft]{threeparttable}


\usepackage{listings}
\usepackage{color}
 
%http://latexcolor.com/ 
\definecolor{codegray}{rgb}{0.5,0.5,0.5}
\definecolor{cerulean}{rgb}{0.0, 0.48, 0.65}
\definecolor{beaublue}{rgb}{0.95, 0.95, 0.95}
\definecolor{amber}{rgb}{1.0, 0.25, 0.0}
\definecolor{indiagreen}{rgb}{0.07, 0.53, 0.03}
\definecolor{number}{rgb}{0.01, 0.01, 0.01}


\lstset{language=C}

  \lstset{emphstyle=\color{blue},
  inputencoding=latin1,
  basicstyle=\footnotesize,
  breaklines=true,
  keywordstyle=\bf\color{amber},
  commentstyle=\color{indiagreen},
  stringstyle=\color{cerulean},
  numberstyle=\color{number},
  backgroundcolor=\color{beaublue},
  morecomment=[s][\color{black}]{[}{]},
  morecomment=[s][\color{cerulean}]{[~}{]:},
  numbers=none,
  numbersep=5pt,
  lineskip=0.7pt,
  columns=fullflexible, % was flexible before but it induce space in '/2entier' like '/2 entier'
  showstringspaces=false ,
  literate=%
         {é}{{\'e}}1}
        
          \newcommand{\FSource}[1]{%
          \lstinputlisting[texcl=true]{#1}
          }

%%%%%%%%%%%% skip an all paragraphe, between this bornes %%%%%%%%%%%%%%%%%%
%\iffalse
%\fi

%%%%%%%%%%%%%%%%%%%%%%%%%%% nouvelles commandes spécifique au doc %%%%%%%%%%%%%%%%%%

\DeclareRobustCommand{\rchi}{{\mathpalette\irchi\relax}}
\newcommand{\irchi}[2]{\raisebox{\depth}{$#1\chi$}}
\renewcommand*\contentsname{Table des matières}
\newcommand{\unit}[1]{\hfill\text{}[\mathrm{#1}]}
%%%%%%%%%%%%%%%%%%%%%%%%%%%%%%%%%%%%%%%%%%%%%%%%%%%%%%%%%%%%%%%%%%%%%%
% Page de garde
%%%%%%%%%%%%%%%%%%%%%%%%%%%%%%%%%%%%%%%%%%%%%%%%%%%%%%%%%%%%%%%%%%%%%%
\begin{document}

\begin{figure}
    \begin{minipage}{.75\textwidth}
    \begin{center}
    {\Large INF203 : Compte-rendu TP6 \\ \textbf{C : Chaines de caracteres, structures}}
    \end{center}
    %\vspace{\baselineskip}
    %\setlength{\parskip}{\smallskipamount}
    \rule{7em}{.4pt}\par
     Alexandre Dupré, Maxime Jaunatre, Clément Raspail | INF - 3 \par 
     %\href{mailto:\href{mailto:maxime.jaunatre@etu.univ-grenoble-alpes.fr}{Mail $^1$} | \today \par 
     \href{mailto:alexandre.dupre@etu.univ-grenoble-alpes.fr,maxime.jaunatre@etu.univ-grenoble-alpes.fr, clement.raspail@etu.univ-grenoble-alpes.fr}{Mail} | \today
\end{minipage}
\end{figure}

\hrule

\section*{Syntaxe}

\iffalse
 Alexandre Dupré <alexandre.dupre@etu.univ-grenoble-alpes.fr>
 Maxime Jaunatre <maxime.jaunatre@etu.univ-grenoble-alpes.fr> 
 Clément Raspail <clement.raspail@etu.univ-grenoble-alpes.fr>
\fi

Pour ce compte rendu la syntaxe des commandes sera la suivante :
\begin{lstlisting}
[~chemin]: commande
retour de la commande
\end{lstlisting}

Example :
\begin{lstlisting}
[~/INF203]: ls
sauve_TP1   TP1   TP2
\end{lstlisting}

Si la commande est interactive et demande d'appuyer  sur entrée, une caractère '->' est indiqué.

Les fichiers sont en \textit{italique} et les commandes (ou détails de retour de commande) en \textbf{gras}. Un script sera donc en gras quand il sera appelé comme une commande.


Les fichiers sources en C sont compilés avec clang, et un nom est donné avec l'option \textbf{clang -o}. Cela implique que les programmes seront appelés par un autre nom que \textbf{a.out}.

% #################################################################################################################################################################

\subsection*{Pour s'échauffer : quelques gammes avec des chaînes de caractères}


[a] 
\begin{lstlisting}
'p' 'l' 'o' 'm' 't' 'r' 'a' 'l' 'a' 'l' 'a' '\0'
\end{lstlisting}


[b]
\begin{lstlisting}
[~/INF203/TP6]: clang -o chaines chaines.c ; ./chaines 
un mot (pas trop long !) à mesurer ?
cthulhu
longueur de la chaine cthulhu :7
On a bien copié ici le mot : cthulhu 
Les deux chaines cthulhu et cthulhu sont identiques
On a bien concaténé ici le mot avec lui même : cthulhucthulhu 
Les deux chaines cthulhucthulhu et cthulhu sont différentes
\end{lstlisting}

\begin{lstlisting}[numbers=left, firstnumber = 0 ]
[~/INF203/TP6]: cat chaines.c
#include <stdio.h>
#include <string.h>
#include "generer_entier.c"
#include "bouts_de_phrases.c"

/* longueur de la chaine passee en parametre */
unsigned long mon_strlen(char *ch) {
	int i;
	for (i=0 ; ch[i] != '\0' ; i++)
		;
	return i ;
}

void mon_strcopy(char *destination, char *source){
	int i;
	for(i=0;source[i]!= '\0'; i++){
		destination[i] = source[i];
	}
	destination[i+1] = '\0';
}

/* Concatène source à la fin de destination */
void mon_strcat(char *destination, char *source){
	int i;
	int j = 0;
	for(i = mon_strlen(destination) ;source[j]!= '\0'; i++){
		destination[i] = source[j];
		j++;
	}
	destination[i+1] = '\0';
}

int mon_strcmp(char *chaine1, char *chaine2){
	int i;
	int rep = 0;
	for(i = 0; chaine1[i]!= '\0' || chaine2[i]!= '\0' ;i++){
		if (chaine1[i] != chaine2[i]){
			rep=1;
		}
	}
	return rep;
}

int main() {
	char chaine[50] ;
	unsigned long mon_resultat ;

	printf("un mot (pas trop long !) à mesurer ?\n") ;
	scanf("%49s", chaine) ;
	mon_resultat=mon_strlen(chaine) ;
	if (mon_resultat == strlen(chaine) )
		printf("longueur de la chaine %s :%lu\n", chaine, mon_resultat) ;
	else
		printf("non, la longueur de '%s' n'est pas %lu\n", chaine, mon_resultat) ;


	char temp[mon_resultat *2 + 1];
	mon_strcopy(temp, chaine);
	printf("On a bien copié ici le mot : %s \n", temp);


	int retour;
	retour=mon_strcmp(temp,chaine);
	if (retour == 0)
		printf("Les deux chaines %s et %s sont identiques\n", temp, chaine);
	else
		printf("Les deux chaines %s et %s sont différentes\n", temp, chaine);


	mon_strcat(temp, chaine);
	printf("On a bien concaténé ici le mot avec lui même : %s \n", temp);
	

	retour=mon_strcmp(temp,chaine);
	if (retour == 0)
		printf("Les deux chaines %s et %s sont identiques\n", temp, chaine);
	else
		printf("Les deux chaines %s et %s sont différentes\n", temp, chaine);
	return 0;
\end{lstlisting}

\textbf{Exercice complémentaire}
\begin{lstlisting}[numbers=left, firstnumber = 33 ]
int mon_strcmp(char *chaine1, char *chaine2){
	int i;
	int rep = 0;
	for(i = 0; chaine1[i]!= '\0' || chaine2[i]!= '\0' ;i++){
		if (chaine1[i] != chaine2[i]){
			if (chaine1[i] < chaine2[i])
				rep = -1;
			else 
				rep = 1;
			return rep;
		}
	}
	return rep;
\end{lstlisting}

\subsection*{Fabriquer des phrases}

[c] 
\begin{lstlisting}[numbers=left, firstnumber = 0 ]
[~/INF203/TP6]: cat phrases.c
#include <stdio.h>
#include <string.h>

int main() {
        char Sujet[50]="la petite souris" ;
        char Verbe[50]="mange" ;
        char Compl[50]="le gros chat" ;
        char Phrase[150] ;

        strcat(Phrase,Sujet);
        strcat(Phrase, " ");
        strcat(Phrase,Verbe);
        strcat(Phrase, " ");
        strcat(Phrase,Compl);
        printf("%s\n", Phrase);
        return 0;
}
\end{lstlisting}

\begin{lstlisting}
[~/INF203/TP6]: clang -o phrases phrases.c ; ./phrases
la petite souris mange le gros chat
\end{lstlisting}

[d]
On compte les espaces pour compter les mots. Il ne faut pas oublier d'initier à 1 pour le premier mot de la phrase.

\begin{lstlisting}[numbers=left, firstnumber = 0 ]
[~/INF203/TP6]: cat phrases.c
#include <stdio.h>
#include <string.h>
#include "bouts_de_phrases.c"
#include "generer_entier.c"


int nb_mots(char *Phrase){
	int i,j = 1;
	for (i=0; Phrase[i] != '\0'; i++){
		if(Phrase[i] == ' ')
			j++;
	}
	return j;
}


int main() {
	char Sujet[50]="la petite souris" ;
	char Verbe[50]="mange" ;
	char Compl[50]="le gros chat" ;
	char Phrase[150] ;
	int res,i;

	strcat(Phrase,Sujet);
	strcat(Phrase, " ");
	strcat(Phrase,Verbe);
	strcat(Phrase, " ");
	strcat(Phrase,Compl);
	// printf("%s\n", Phrase);

	for(i = 0; i < 3; i++){
		Phrase[0] = '\0'; // reinitialise phrase
		strcat(Phrase,sujet[generer_entier(11)]);
		strcat(Phrase, " ");
		strcat(Phrase,verbe[generer_entier(11)]);
		strcat(Phrase, " ");
		strcat(Phrase,complement[generer_entier(11)]);
		printf("%s\n", Phrase);
		res = nb_mots(Phrase);
		printf("La phrase : %s contient %d mots\n", Phrase, res);
	}
	return 0;
}
\end{lstlisting}

\begin{lstlisting}
[~/INF203/TP6]: clang -o phrases phrases.c ; ./phrases
Guillaume H. détruit son steack-frites
La phrase : Guillaume H. détruit son steack-frites contient 5 mots
un inconnu avale bruyamment la fumee qui sort du clavier
La phrase : un inconnu avale bruyamment la fumee qui sort du clavier contient 10 mots
Oscar trouve enfin la fumee qui sort du clavier
La phrase : Oscar trouve enfin la fumee qui sort du clavier contient 9 mots
\end{lstlisting}


[e]
Cela n'aurait pas marché car on ne pointe pas joueur directement, il faut donc préciser (joueur* j) pour que cela fonctionne 
correctement.


[f]
\begin{lstlisting}[numbers=left, firstnumber = 0 ]
[~/INF203/TP6]: cat billes.c
#include <stdio.h>
#include <string.h>

typedef struct {
        char pseudo[20];
        int nb_billes;
} joueur;

joueur atchoum = { "Atchoum", 42 };
joueur dormeur = { "Dormeur", 25 };
joueur grincheux = { "Grincheux", 3 };
joueur joyeux = { "Joyeux", 100 };
joueur prof = { "Prof", 2 };
joueur simplet = { "Simplet", 0 };
joueur timide = { "Timide", 12 };

void afficher_joueur_V1(joueur j) {
        printf("%s a %d billes\n", j.pseudo, j.nb_billes);
}

void afficher_joueur_V2(joueur* pj) {
        printf("%s a %d billes\n", pj->pseudo, pj->nb_billes);
}

void lire_joueur(joueur* pj){
        char t_pseudo[20] ;
        int nbille ;
        scanf("%19s", t_pseudo) ;
        scanf("%d", &nbille) ;
        strcpy(pj->pseudo,t_pseudo);
        pj->nb_billes = nbille;
}

void transferer_V1(joueur j1, joueur j2, int nb){
        j1.nb_billes = j1.nb_billes - nb;
        j2.nb_billes = j2.nb_billes + nb;
}

void transferer_V2(joueur* j1, joueur* j2, int nb){
        j1->nb_billes = j1->nb_billes - nb;
        j2->nb_billes = j2->nb_billes + nb;
}


int main() {
        afficher_joueur_V1(atchoum);
        afficher_joueur_V2(&simplet);
        lire_joueur(&simplet);
        afficher_joueur_V2(&simplet);
        printf("\nOn test transferer_V1 avec 10 billes\n");
        transferer_V1(atchoum,simplet,10);
        afficher_joueur_V2(&atchoum);
        afficher_joueur_V2(&simplet);
        printf("\nOn test transferer_V2 avec 10 billes\n");
        transferer_V2(&atchoum,&simplet,10);
        afficher_joueur_V2(&atchoum);
        afficher_joueur_V2(&simplet);
        return 0;
}
\end{lstlisting}


\begin{lstlisting}
[~/INF203/TP6]: clang -o billes billes.c ; ./billes
Atchoum a 42 billes
Simplet a 0 billes
Roger
1
Roger a 1 billes

On test transferer_V1 avec 10 billes
Atchoum a 42 billes
Roger a 1 billes

On test transferer_V2 avec 10 billes
Atchoum a 32 billes
Roger a 11 billes
\end{lstlisting}

\newpage

\subsection*{Commande de la semaine : \textbf{tr}}

[g]
\textit{On a pris la liberté de modifier légèrement exemple\_tr...}
\begin{lstlisting}
[~/INF203/TP6]: tr aeiou vwxyz < exemples_tr 
Ww'rw ny strvngwrs ty lyvw
Yyz knyw thw rzlws vnd sy dy I
A fzll cymmxtmwnt's whvt I'm thxnkxng yf
Yyz wyzldn't gwt thxs frym vny ythwr gzy

I jzst wvnnv twll yyz hyw I'm fwwlxng
Gyttv mvkw yyz zndwrstvnd

Nwvwr gynnv gxvw yyz zp
Nwvwr gynnv lwt yyz dywn
Nwvwr gynnv rzn vryznd vnd dwswrt yyz
Nwvwr gynnv mvkw yyz cry
Nwvwr gynnv svy gyydbyw
Nwvwr gynnv twll v lxw vnd hzrt yyz
\end{lstlisting}

La commande utilisée est:
\begin{lstlisting}
[~/INF203/TP6]: cat exemples_tr | tr -s '\n' > fich2 ; cat fich2
We're no strangers to love
You know the rules and so do I
A full commitment's what I'm thinking of
You wouldn't get this from any other guy
I just wanna tell you how I'm feeling
Gotta make you understand
Never gonna give you up
Never gonna let you down
Never gonna run around and desert you
Never gonna make you cry
Never gonna say goodbye
Never gonna tell a lie and hurt you
\end{lstlisting}


%%%%%%%%%%%%%%%%%%%%%%%%%%%%%%%%%%%%%%%%%%%%%%%%%%%%%%%%%%%%%%%%%%%%%%
% Références
%%%%%%%%%%%%%%%%%%%%%%%%%%%%%%%%%%%%%%%%%%%%%%%%%%%%%%%%%%%%%%%%%%%%%%
%\newpage
\cite{}
%\subsection*{Bibliographie}
\bibliographystyle{authordate1}
\bibliography{ICU}

\end{document}